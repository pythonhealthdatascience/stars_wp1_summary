% Options for packages loaded elsewhere
\PassOptionsToPackage{unicode}{hyperref}
\PassOptionsToPackage{hyphens}{url}
%
\documentclass[
]{article}
\usepackage{lmodern}
\usepackage{amssymb,amsmath}
\usepackage{ifxetex,ifluatex}
\ifnum 0\ifxetex 1\fi\ifluatex 1\fi=0 % if pdftex
  \usepackage[T1]{fontenc}
  \usepackage[utf8]{inputenc}
  \usepackage{textcomp} % provide euro and other symbols
\else % if luatex or xetex
  \usepackage{unicode-math}
  \defaultfontfeatures{Scale=MatchLowercase}
  \defaultfontfeatures[\rmfamily]{Ligatures=TeX,Scale=1}
\fi
% Use upquote if available, for straight quotes in verbatim environments
\IfFileExists{upquote.sty}{\usepackage{upquote}}{}
\IfFileExists{microtype.sty}{% use microtype if available
  \usepackage[]{microtype}
  \UseMicrotypeSet[protrusion]{basicmath} % disable protrusion for tt fonts
}{}
\makeatletter
\@ifundefined{KOMAClassName}{% if non-KOMA class
  \IfFileExists{parskip.sty}{%
    \usepackage{parskip}
  }{% else
    \setlength{\parindent}{0pt}
    \setlength{\parskip}{6pt plus 2pt minus 1pt}}
}{% if KOMA class
  \KOMAoptions{parskip=half}}
\makeatother
\usepackage{xcolor}
\IfFileExists{xurl.sty}{\usepackage{xurl}}{} % add URL line breaks if available
\IfFileExists{bookmark.sty}{\usepackage{bookmark}}{\usepackage{hyperref}}
\hypersetup{
  hidelinks,
  pdfcreator={LaTeX via pandoc}}
\urlstyle{same} % disable monospaced font for URLs
\usepackage{longtable,booktabs}
% Correct order of tables after \paragraph or \subparagraph
\usepackage{etoolbox}
\makeatletter
\patchcmd\longtable{\par}{\if@noskipsec\mbox{}\fi\par}{}{}
\makeatother
% Allow footnotes in longtable head/foot
\IfFileExists{footnotehyper.sty}{\usepackage{footnotehyper}}{\usepackage{footnote}}
\makesavenoteenv{longtable}
\setlength{\emergencystretch}{3em} % prevent overfull lines
\providecommand{\tightlist}{%
  \setlength{\itemsep}{0pt}\setlength{\parskip}{0pt}}
\setcounter{secnumdepth}{-\maxdimen} % remove section numbering

\author{}
\date{}

\begin{document}

\section{STARS Framework}

\subsection{Checklist}

\begin{longtable}[]{@{}ll@{}}
\toprule
Item & Implementation\tabularnewline
\midrule
\endhead
\textbf{Essential components} &\tabularnewline
Open licence & \emph{Describe your implementation of this
component}\tabularnewline
Dependency management &\tabularnewline
FOSS model &\tabularnewline
Minimum documentation &\tabularnewline
ORCID &\tabularnewline
Citation information &\tabularnewline
Remote code repository &\tabularnewline
Open science archive &\tabularnewline
\textbf{Optional components} &\tabularnewline
Enhanced documentation &\tabularnewline
Documentation hosting &\tabularnewline
Online coding environment &\tabularnewline
Model interface &\tabularnewline
Web app hosting &\tabularnewline
\bottomrule
\end{longtable}

\subsection{Description}

\subsubsection{Essential components}

\paragraph{Open licence}

Free and open-source software (FOSS) licence (e.g. MIT, GNU Public
Licence (GPL))

\paragraph{Dependency management}

Specify software libraries, version numbers and sources (e.g. dependency
management tools like virtualenv, conda, poetry)

\paragraph{FOSS model}

Coded in FOSS language (e.g. R, Julia, Python)

\paragraph{Minimum documentation}

Minimal instructions (e.g. in README) that overview (a) what model does,
(b) how to install and run model to obtain results, and (c) how to vary
parameters to run new experiments

\paragraph{ORCID}

ORCID for each study author

\paragraph{Citation information}

Instructions on how to cite the research artefact (e.g. CITATION.cff
file)

\paragraph{Remote code repository}

Code available in a remote code repository (e.g. GitHub, GitLab,
BitBucket)

\paragraph{Open science archive}

Code stored in an open science archive with FORCE11 compliant citation
and guaranteed persistance of digital artefacts (e.g. Figshare, Zenodo,
the Open Science Framework (OSF), and the Computational Modeling in the
Social and Ecological Sciences Network (CoMSES Net))

\subsubsection{Optional components}

\paragraph{Enhanced documentation}

Open and high quality documentation on how the model is implemented and
works (e.g. via notebooks and markdown files, brought together using
software like Quarto and Jupyter Book). Suggested content includes:

\begin{itemize}
\tightlist
\item
  Plain english summary of project and model
\item
  Clarifying licence
\item
  Citation instructions
\item
  Contribution instructions
\item
  Model installation instructions
\item
  Structured code walk through of model
\item
  Documentation of modelling cycle using TRACE
\item
  Annotated simulation reporting guidelines
\item
  Clear description of model validation including its intended purpose
\end{itemize}

\paragraph{Documentation hosting}

Host documentation (e.g. with GitHub pages, GitLab pages, BitBucket
Cloud, Quarto Pub)

\paragraph{Online coding environment}

Provide an online environment where users can run and change code (e.g.
BinderHub, Google Colaboratory, Deepnote)

\paragraph{Model interface}

Provide web application interface to the model so it is accessible to
less technical simulation users

\paragraph{Web app hosting}

Host web app online (e.g. Streamlit Community Cloud, ShinyApps hosting)

\end{document}
